\documentclass[a4paper]{article}

% 边距
\usepackage[margin=0.72in]{geometry}

\usepackage{float}
% 图片格式
\usepackage{subfigure}
\usepackage{graphicx}

% 代码块格式
\usepackage{listings}
\usepackage{xcolor}
\usepackage{fontspec}
\newfontfamily\menlo{Menlo}
\lstset{
 columns=fixed,       
 numbers=left,                                        % 在左侧显示行号
 numberstyle=\tiny\menlo,	                          % 设定行号格式
 basicstyle=\small\menlo,							  % 字体
 frame=none,                                          % 不显示背景边框
 backgroundcolor=\color[RGB]{245,245,244},            % 设定背景颜色
 keywordstyle=\color[RGB]{40,40,255},                 % 设定关键字颜色
 numberstyle=\footnotesize\color{darkgray},           
 commentstyle=\color[RGB]{0,96,96},                % 设置代码注释的格式
 stringstyle=\rmfamily\slshape\color[RGB]{128,0,0},   % 设置字符串格式
 stringstyle=\color[RGB]{128,0,0},   				  % 设置字符串格式
 showstringspaces=false,                            % 不显示字符串中的空格
}

% used for block comment
\newcommand{\eat}[1]{}

\usepackage{cite}

\title{VE482 Lab9}
% \author{Shen Feiyu 517370910116}

\begin{document}

\maketitle

\section{Read Write Return Value}
The read operation should copy the desired data from kernel space to user space 
by functions like put\_user. \\
The write operation should copy the data from user space buffer to kernel space 
and do some operation. It's done through functions like get\_user.\\

\section{Major Number and Minor Number}
The major number tells you which driver handles which device file. 
The minor number is used only by the driver itself to differentiate which device it's operating on, just in case the driver handles more than one device.

\section{Add Char Device}
\begin{lstlisting}[language=c]
// cdev variable
static struct cdev diceDev0;
// init cdev with customized file operation
cdev_init(&diceDev0, &fops0);
// add device with major number and minor number in MKDEV(majorNumber, minorNumber) 
// (here minorNumber is 0)
cdev_add(&diceDev0, MKDEV(majorNumber, 0), 1);
\end{lstlisting}

\section{Code Location}
\begin{lstlisting}[language=c]
// module_init: /usr/src/linux-headers-5.4.0-53/include/linux/module.h
// module_exit: /usr/src/linux-headers-5.4.0-53/include/linux/module.h
// printk:      /usr/src/linux-headers-5.4.0-53/include/linux/printk.h
// container_of: /usr/src/linux-headers-5.4.0-53/include/linux/kernel.h
// dev_t:         /usr/src/linux-headers-5.4.0-53/include/linux/device.h

// MAJOR:     /usr/src/linux-headers-5.4.0-53/include/uapi/linux/kdev_t.h
// MINOR:     /usr/src/linux-headers-5.4.0-53/include/uapi/linux/kdev_t.h
// MKDEV:     /usr/src/linux-headers-5.4.0-53/include/linux/kdev_t.h
// alloc_chrdev_region:  /usr/src/linux-headers-5.4.0-53/include/linux/fs.h
// module_param:    /usr/src/linux-headers-5.4.0-53/include/linux/moduleparam.h
// cdev_init:        /usr/src/linux-headers-5.4.0-42/include/linux/cdev.h
// cdev_add:       /usr/src/linux-headers-5.4.0-42/include/linux/cdev.h
// cdev_del:       /usr/src/linux-headers-5.4.0-42/include/linux/cdev.h
// THIS_MODULE:   /usr/src/linux-headers-5.4.0-42/include/linux/export.h:
\end{lstlisting}

\section{Generate Random Number}
\begin{lstlisting}[language=c]
static struct timespec ts;

static int gen_rand(int mod) {
    int rd;
    getnstimeofday(&ts);
    rd = ts.tv_nsec % mod;
    return abs(rd);
}
\end{lstlisting}


\section{Defind and Use Mudule Options}
\begin{lstlisting}[language=c]
// in code, define and use gen_sides like common variable
static int gen_sides = 6;
module_param(gen_sides,int,S_IRUGO);
// when insmod, use:
// sudo insmod dice.ko gen_sides=20
\end{lstlisting}

\section{Result}

\begin{lstlisting}[language=sh]
francis@ubuntu:~/code/lab9$ make
make -C /lib/modules/5.4.0-53-generic/build/ M=/home/francis/code/lab9 modules CFLAGS='-std=c11'
make[1]: Entering directory '/usr/src/linux-headers-5.4.0-53-generic'
    CC [M]  /home/francis/code/lab9/dice.o
    Building modules, stage 2.
    MODPOST 1 modules
    LD [M]  /home/francis/code/lab9/dice.ko
make[1]: Leaving directory '/usr/src/linux-headers-5.4.0-53-generic'
francis@ubuntu:~/code/lab9$ sudo insmod dice.ko gen_sides=20
[sudo] password for francis:
francis@ubuntu:~/code/lab9$ sudo -i
root@ubuntu:~# echo 1 > /dev/DiceDev0
root@ubuntu:~# cat /dev/DiceDev0
-------
| o o |
|  o  |
| o o |
-------
root@ubuntu:~# echo 2 > /dev/DiceDev1
root@ubuntu:~# cat /dev/DiceDev1
2 5
root@ubuntu:~# echo 3 > /dev/DiceDev2
root@ubuntu:~# cat /dev/DiceDev2
15 2 6
root@ubuntu:~# ls -l /dev/ | grep Dice
crw-------  1 root    root    507,   0 Nov 25 22:42 DiceDev0
crw-------  1 root    root    507,   1 Nov 25 22:42 DiceDev1
crw-------  1 root    root    507,   2 Nov 25 22:42 DiceDev2
\end{lstlisting}

% \newpage
% % reference part
% \renewcommand\refname{Reference}
% \bibliographystyle{plain}
% \bibliography{ref}         % reference bib file

\end{document}